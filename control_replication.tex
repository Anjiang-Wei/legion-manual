\chapter{Control Replication}
\label{chap:cntrlrep}


\begin{figure}
  {\small
    \lstinputlisting[linerange={15-44,59-74}]{Examples/ControlReplication/sum/cp.cc}
  }
  \caption{\legionbook{ControlReplication/sum/cp.cc}}
  \label{fig:ctrlrep}
\end{figure}

In Legion and most other tasking models, a root task is responsible
for launching subtasks that will fill a parallel machine.  We have
already seen that index launches (recall
Section~\ref{sec:indexlaunch}) can be used to compactly express the
launch of a set of $n$ tasks, where $n$ is usually scaled with the
size of the machine being used.

A problem arises when the number of child tasks to be launched by a parent task
is large: The amount of work the parent task needs to do to launch all of the child tasks
can itself become a serial bottleneck in the program.  In practice, it turns out that this
effect does not require especially large numbers of tasks to become noticeable.  For most
applications, a parent task repeatedly launching more than 16 or 32 tasks at a time has
a measurable impact on scalability.

{\em Control replication} is Legion's solution to this problem and a key
feature of the programming model.  Almost any application with tasks
that launch a large number of subtasks will perform
significantly better with control replication.  

The idea behind control replication is simple: Instead of having one copy of the
parent task launching all of the child tasks, multiple copies of the parent
task are executed in parallel, each of which launches a subset of the child tasks.
For example, if a parent tasks launches subtasks using index launches,
then control-replicating the parent tasks $n$ times will result in all copies of the
parent task launching $1/n$th of the tasks in each index launch (using the default mapper,
see below).  A common pattern is to replicate a task once per Legion process
in the computation, with each replicated instance launching the subtasks 
destined to execute locally on the resources managed by that Legion runtime.

By far the most common case
is that the top-level task is control-replicated and all other tasks are not,
but sometimes overall performance can be improved by control replicating other
tasks in the task hieararchy.  It is also legal to nest control-replicated tasks:
control-replicated tasks can be launched from within other control-replicated tasks.

At the program level, the use of control replication is straightforward.  Typically,
the only thing that needs to be done is to notify the system that a task is {\em replicable},
as shown in Figure~\ref{fig:ctrlrep}, line 36.  In this example, the top-level
task is marked as replicable, while the {\tt sum} task (not shown) is not.
As we discuss below, not every task is replicable.  Even if a task $t$ is potentially
replicable, if it does not launch enough subtasks to make control replication
worthwhile then it will be better overall not to replicate $t$.

If a task is marked as replicable, then the decisions of whether to
replicate the task or not and, if the task is replicated, how to {\em
  shard} the work of analyzing the subtasks across all the instances
of the replicated task are made by the mapper.  The default mapper
handles the case of a top-level task that is control-replicated and
subtasks are sharded evenly and to the instance of the Legion runtime
where they will execute.  Anything other than this simple, but very
common, case will likely require writing some custom mapping
logic.  Because the dynamic safety checks for control
replication do not currently cover every way that a task
might fail to be replicable, it is possible, but unlikely, for a task that is
incorrectly marked as replicable to pass the safety checks.


    
