\chapter{Tasks}
\label{chap:tasks}

% Set up the listing configuration for everything
\lstset{
  captionpos=b,
  language=C++,
  basicstyle=\scriptsize,
  numbers=left,
  numberstyle=\tiny,
  columns=fullflexible,
  stepnumber=1,
  escapechar=\#,
  keepspaces=true,
  literate={<}{{$\langle$}}1 {>}{{$\rangle$}}1,
  %xleftmargin=15pt,
}

The Legion runtime is a \Cpp\ library, and
Legion programs are just C++ programs that use the Legion runtime API.
One important consequence of this design is that almost all Legion decisions
(such as what data layout to use, in which memories to place data and on which
processors to run computations) are made dynamically,  during the execution of 
a Legion application.  Dynamic decision making provides maximum flexibility, 
allowing the runtime's decisions to be reactive to the current state of the computation.
Implementing Legion as a \Cpp\ library also allows high performance \Cpp\ code
(e.g., vectorized kernels) to be used seamlessly in Legion applications.

In Legion, {\em tasks} are distinguished functions with a specific signature.
Legion tasks have several important properties:
\begin{itemize}
\item Tasks are the unit of parallelism in Legion; all parallelism occurs because tasks are executed in parallel.

\item Tasks are specific to a particular kind of {\em processor} (most commonly CPUs or GPUs), but a task can always be run on any processor
of the correct kind.

\item Once a processor is picked for a task, that task will execute in its entirety on that processor---tasks do
not migrate.  

\end{itemize}


\begin{figure}
\begin{lstlisting}
#include <cstdio>
#include "legion.h"

using namespace Legion;

// All tasks must have a unique task id (a small integer).
// A global enum is a convenient way to assign task ids.
enum TaskID {
  SUM_ID,
};

void sum_task(const Task *task,
              const std::vector<PhysicalRegion> &regions,
              Context ctx, 
              Runtime *runtime)
{
  int sum = 0;
  for (int i = 0; i <= 1000; i++) {
    sum += i;
  }
  printf("The sum of 0..1000 is %d\n", sum);
}

int main(int argc, char **argv)
{
  Runtime::set_top_level_task_id(SUM_ID);
  Runtime::register_legion_task<sum_task>(
        SUM_ID,
	Processor::LOC_PROC, 
	true/*single launch*/, 
	false/*no multiple launch*/);
  return Runtime::start(argc, argv);
}
\end{lstlisting}
\caption{Tasks/sum/sum.cc}
\label{fig:simple}
\end{figure}

Figure~\ref{fig:simple} shows a very simple, but complete, Legion program for summing
the first 1000 positive integers (also available as {\tt sum.cc} in \legionbook{Tasks}).  
At a high level, every Legion program has three components:
\begin{itemize}
\item The id of the top-level task must be set with Legion's {\em high level runtime}.  The top-level
task is the initial task that is called when the Legion runtime starts.
\item Every task and its task id must be registered with the high level runtime.  Currently all tasks
must be registered before the runtime starts.
\item The start method of the high level runtime is invoked, which in turn calls the top-level task.
Note that by default this call does not return---the program is terminated when the start method terminates.
\end{itemize}
In Figure~\ref{fig:simple}, these three steps are the three statements of {\tt main}.  
The only task in this program is {\tt sum\_task}, which is also the top-level task invoked when the
Legion runtime starts up.  Note that the program does not say where the task is executed; that decision is made
at runtime by the {\em mapper} (see Chapter~\ref{chap:mapping}).  Note also that tasks can perform almost arbitrary
\Cpp\ computations.  In the case of {\tt sum\_task}, the computation performed is very simple, but in general tasks
can call ordinary \Cpp\ functions, including allocating and deallocating memory.  Tasks must not, however,
call directly into other packages that provide parallelism or concurrency.  Interoperation with MPI is
possible but must be done in a standardized way (see Chapter~\ref{chap:interop}).  

As mentioned above, every task must be registered with the Legion runtime before
the runtime's {\tt start} method is called.  Registration passes several arguments about a
task to the runtime:
\begin{itemize}
\item The name of the subtask is a template argument to the {\tt register\_legion\_task} method.

\item The task ID is the first (regular) argument.

\item The kind of processor the task can run on is the second argument.  Current options are
{\em latency optimized cores} or CPUs (constant {\tt LOC}) and {\em throughput optimized cores} or GPUs
(constant {\tt TOC}).

\item Two boolean flags, the first of which indicates whether the task can be used in a single task
launch and the second of which indicates whether the task can be used in a multiple (or {\em index}) task
launch.
\end{itemize}

We will see shortly that tasks can call other tasks and pass
those tasks arguments and return results.  Because the called task may
be executed in a different address space than the caller, arguments
passed between tasks must not contain \Cpp\ pointers, as these will
not make sense outside of the address space in which they were
created.  Neither should tasks refer to global variables. A common
programming error for beginning Legion programmers is to pass
\Cpp\ pointers or references between tasks, or to refer to global
variables from within tasks.  As long as all the tasks are mapped to a
single node (i.e., the same address space) the program is likely to
work, but when efforts are made to scale up the application by running
on multiple nodes, \Cpp\ crashes result from the wild pointers or
references to distinct instances of global variables of the same name
in different address spaces.  It is possible to pass data structures
between tasks, but not by using \Cpp\ pointers (see
Chapter~\ref{chap:regions}).

All tasks have the same input signature as {\tt sum\_task}:
\begin{itemize}

\item {\tt const Task *task}: An object representing the task itself. 

\item {\tt const std::vector<PhysicalRegion> \&regions}: A vector of {\em physical region instances}.  This argument is the
primary way to pass data between tasks (see Chapter~\ref{chap:regions}).

\item {\tt Context ctx}: Every task is called in a context, which contains metadata for the task.  Application programs
should not directly manipulate the context.

\item {\tt Runtime *runtime}: A pointer to the runtime, which gives the task access to the Legion runtime's methods.

\end{itemize}

\section{Subtasks}
\label{sec:subtasks}


\begin{figure}
\begin{lstlisting}
void top_level_task(const Task *task,
                    const std::vector<PhysicalRegion> &regions,
                    Context ctx, 
                    Runtime *runtime)
{
  printf("Top level task start.\n");
  for(int i = 1; i <= 100; i++) {
    TaskLauncher launcher(SUBTASK_ID, TaskArgument(&i,sizeof(int)));
    runtime->execute_task(ctx,launcher);
  }
  printf("Top level task ddone launching subtasks.\n");
}

void subtask(const Task *task,
             const std::vector<PhysicalRegion> &regions,
             Context ctx,
             Runtime *runtime)
{
  int subtask_number = *((int *) task->args);
  printf("\tSubtask %d\n", subtask_number);
}
\end{lstlisting}
\caption{\legionbook{Tasks/subtask/subtask.cc}}
\label{fig:subtask}
\end{figure}



Task can call other tasks, known as {\em subtasks}.  We also refer to
the calling task as the {\em parent task} and the called task as the
{\em child task}.  Two or more child tasks of the same parent task are
{\em sibling tasks}.  Figure~\ref{fig:subtask} shows the definition of
the parent task and the child task from the example
\legionbook{Tasks/subtask/subtask.cc}.

Consider the parent task {\tt top\_level\_task}.  There are two steps
to executing a subtask.  First, a {\tt TaskLauncher} object is
created.  The {\tt TaskLauncher}
constructor takes two arguments, the ID of the task to be called and a
{\tt TaskArgument} object that holds a pointer to a buffer containing
data for the subtask together with the size of the buffer.  The
semantics of the task arguments are particularly important.  Recall
that a task may be run on any processor in the system (of a kind that
can execute the task).  Thus, the parent task and the child task may
run in different address spaces, and so the arguments are passed
{\em by value}, meaning that the buffer pointed to by the {\tt TaskArgument} is
copied to where the subtask runs.  Even if the subtask happens to run in the
same address space as the parent task, the buffer referenced by the {\tt
  TaskArgument} is passed by value (i.e., copied).  

{\tt TaskArgument} objects should be used to pass small amounts of data,
such as an integer, float, struct or a (very) small array.  To pass large amounts of
data, use regions (see Chapter~\ref{chap:regions}).  As
discussed earlier in this chapter, task arguments may not contain
\Cpp\ pointers or references.  In addition, task arguments may not contain
futures (see Section~\ref{sec:futures}).

A subtask is actually launched by the {\tt runtime->execute\_task}
method, which requires both the parent task's context and the {\tt
  TaskLauncher} object for the subtask as arguments.  Note that the
the argument buffer pointed to by the {\tt TaskArgument} is copied
only when {\tt execute\_task} is called. On the callee's side, note
that the task arguments are available as a field of the {\tt task}
object. Since \Cpp\ doesn't know the type of the buffer, it is
necessary to first cast the pointer to the buffer to the correct type
before it can be used.

Finally, there are two other important properties of subtasks.  First,
the {\tt execute\_task} method is {\em non-blocking}, meaning it
returns immediately and the subtask is executed asynchronously from
the parent task, allowing the parent task to continue executing while
the subtask is running (potentially) in parallel.  In {\tt
  subtask.cc}, the parent task launches all of the subtasks in a loop,
sending each subtask a unique integer argument that the subtask simply prints
out.  Compile and run {\tt subtask.cc} and observe that the
parent task reports that it is done launching all of the subtasks
before all of the subtasks execute.  Second, parent tasks do not
terminate until all of the child tasks have terminated.  Thus, even
though {\tt top\_level\_task} reaches the end of its function body
before all of its child tasks have completed, at that point the parent
task waits until all the child tasks terminate, at which point
{\tt top\_level\_task} itself terminates.

\section{Futures}
\label{sec:futures}

\begin{figure}
\begin{lstlisting}
void top_level_task(const Task *task,
                    const std::vector<PhysicalRegion> &regions,
                    Context ctx,
                    Runtime *runtime)
{
  printf("Top level task start.\n");
  for(int i = 1; i <= 100; i += 2) {
    TaskLauncher producer_launcher(SUBTASK_PRODUCER_ID, 
                                   TaskArgument(&i,sizeof(int)));
    Future doubled_task_number =  
            runtime->execute_task(ctx,producer_launcher);
    TaskLauncher consumer_launcher(SUBTASK_CONSUMER_ID, 
                                   TaskArgument(NULL,0));
    consumer_launcher.add_future(doubled_task_number);
    runtime->execute_task(ctx,consumer_launcher);
  }
  printf("Top level task done launching subtasks.\n");
}

int subtask_producer(const Task *task,
                      const std::vector<PhysicalRegion> &regions,
                      Context ctx,
                      Runtime *runtime)
{
  int subtask_number = *((int *) task->args);
  printf("\tProducer subtask %d\n", subtask_number);
  return subtask_number + 1;
}

void subtask_consumer(const Task *task,
                     const std::vector<PhysicalRegion> &regions,
                     Context ctx,
                     Runtime *runtime)
{
  Future f = task->futures[0];
  int subtask_number = f.get_result<int>();
  printf("\tConsumer subtask %d\n", subtask_number);
}
\end{lstlisting}
\caption{\legionbook{Tasks/futures/futures.cc}}
\label{fig:futures}
\end{figure}

In addition to taking arguments, subtasks may also return results.
However, because a subtask executes asynchronously from its parent
task, there is no guarantee that the result of the subtask will be
available when the parent task or another task attempts to use it.  A
standard solution to this problem is to provide {\em futures}.  A future
is a value that, if read, causes the task that is performing the
read to block if necessary until the value is available.

Figure~\ref{fig:futures} shows an excerpt from {\tt futures.cc}, which
is an extension of {\tt substask.cc} from
Section~\ref{sec:subtasks}.  In this example, there are two subtasks,
a producer and a consumer.  The top level task repeatedly calls
\mbox{producer/consumer} pairs in a loop.  The top level task first calls the
producer task, passing it a unique odd integer, which the producer
prints out.  The producer returns a unique even integer as a future.
The top level task then passes this future to a consumer task that
reads and prints the number.

The launch of the producer task is exactly as before in
Figure~\ref{fig:subtask}.  Unlike in that example, however, the
producer subtask has a non-void return value, and so the {\tt
  runtime->execute\_task} invocation returns a useful result of type
{\tt Future}.  Note that the future is passed to the consumer task
using the {\tt add\_future} method of the {\tt TaskLauncher} class, not
through the {\tt TaskArgument} object used to construct the {\tt
  TaskLauncher}; futures must always be passed as arguments using {\tt add\_future}
and must not be included in {\tt TaskArgument}s.  Having a distinguished method for tracking arguments
to tasks that are futures allows the Legion runtime to track 
{\em dependencies} between tasks.  In this case, the Legion runtime 
will know that the consumer task depends on the result of the corresponding
producer task.

Legion gives access to the value of a future through the {\tt
  get\_result} method of the {\tt Future} class, as shown in the code
for {\tt subtask\_consumer} in Figure~\ref{fig:futures}.  (Note that
{\tt get\_result} is templated on the type of value the future holds.)
There are two interesting cases of tasks reading from futures:
\begin{itemize}

\item If a parent task attempts to access a future returned
by one of its child tasks that has not yet completed, the parent task
will block until the value of the future is available.  This behavior is the
standard semantics for futures, as described above.  In Legion, however,
this style of programming is discouraged, as blocking operations are 
generally detrimental to achieving the highest possible performance.

\item Figure~\ref{fig:futures} illustrates idiomatic use of futures in Legion:
a future returned by one subtask is passed as an argument to another subtask.
Because Legion knows the consumer task depends on the producer task, the consumer
task will not be run by the Legion runtime until the producer task has terminated.
Thus, all references to the future in the consumer task are guaranteed to 
return immediately, without blocking.

\end{itemize}

\section{Points, Rectangles and Domains}

Up to this point we have discussed individual tasks.  Legion also provides mechanisms for
naming and launching sets of tasks.  The ability to name and manipulate sets of things, and 
in particular sets of points, is useful for
more than dealing with sets of tasks, and so we first present the general mechanism in
Legion for defining {\em points}, {\em rectangles} and {\em domains}.

A {\em point} is an n-tuple of integers.  The {\tt Point} constructor, which is templated on the
dimension $n$,  is used to create points:
\begin{verbatim}
Point<1> one(1);                     // The 1 dimensional point <1>
Point<1> two(2);                     // The 1 dimensional point <2>
Point<1> anotherone = make_point(1); // The 1 dimensional point <1>
Point<2> zeroes = make_point(0,0);   // The 2 dimensional point <0,0>
Point<2> twos = make_point(2,2);     // The 2 dimensional point <2,2>
Point<2> threes = make_point(3,3);   // The 2 dimensional point <3,3>
Point<3> fours = make_point(4,4,4);  // The 3 dimensional point <4,4,4>
\end{verbatim}

There are many operations defined on points.  For example, points can be summed:
\begin{verbatim}
twos + threes  // the point <5,5>
\end{verbatim}
and one can take the dot product of two points:             
\begin{verbatim}
twos.dot(threes)  // the integer 12
\end{verbatim}
All of the usual boolean operations are defined on points.  The following are all true:
\begin{verbatim}
twos == twos   
twos != threes 
twos <= threes 
\end{verbatim}
For \verb+<=+, the ordering is lexicographic: a point $a$ is less than or equal to a point $b$ if they either agree
in all coordinates, or the first coordinate in which they disagree reading from the left is smaller in $a$ than in $b$.

A pair of points $a$ and $b$ defines a {\em rectangle} that includes all the points that are greater than or equal to $a$
and less than or equal to $b$.  For example:
\begin{verbatim}
// the points  <0,0> <0,1> <0,2> <0,3> 
//             <1,0> <1,1> <1,2> <1,3>
//             <2,0> <2,1> <2,2> <2,3>
//             <3,0> <3,1> <3,2> <3,3>
Rect<2> big(zeroes,threes);  

// the points  <2,2> <2,3>
//             <3,2> <3,3>
Rect<2> small(twos,threes);
\end{verbatim}
There are also many operations defined on rectangles.  A few examples, all of which evaluate to true:
\begin{verbatim}
big != small                       
big.contains(small)                
small.overlaps(big)                
small.convex_hull(big) == big      
small.intersection(big) == small   
\end{verbatim}
Note that the intersection of two rectangles is always a rectangle.  The union of two or more rectangles is not necessarily a 
rectangle, however. The {\em convex hull} of a set of rectangles $S$ is the smallest rectangle that contains all of the rectangles
in $S$.

A {\em domain} is alternative type for rectangles.  A {\tt Rect} can be converted to a {\tt Domain}:
\begin{verbatim}
Domain bigdomain = Domain::from_rect<2>(big);
\end{verbatim}
The difference between the two types is that {\tt Rect}s are templated on the dimension of the rectangle, while {\tt Domains}
are not.  Legion runtime methods generally take {\tt Domain} arguments and use {\tt Domain}s internally, but for application
code the extra type checking provided by the {\tt Rect} type (which ensures that the operations are applied to {\tt Rect} arguments
with compatible dimensions) is useful.  The recommended programming style is to create {\tt Rect}s and convert them to {\tt Domain}s
at the point of a Legion runtime call.  It is also possible to work directly with the {\tt Domain} type, which has many
of the same methods as {\tt Rect} (see {\tt lowlevel.h} in the {\tt runtime/} directory).

Analagous to {\tt Rect} and {\tt Domain}, there is a less-typed version of the type {\tt Point} called {\tt DomainPoint}.
Again, the difference between the two types is that the {\tt Point} class is templated on the number of dimensions 
while {\tt DomainPoint} is not.  For Legion methods that require a {\tt DomainPoint}, there is a function to convert a
{\tt Point}:
\begin{verbatim}
DomainPoint dtwos = DomainPoint::from_point<2>(twos);
\end{verbatim}
As before, most Legion runtime calls take {\tt DomainPoints}, but programmers should probably prefer using the {\tt Point} type
for the extra type checking provided.

The example program \legionbook{Tasks/domains/domains.cc} includes all of the examples in this section and more.

\section{Index Launches}
\label{sec:indexlaunch}

We now return to the Legion mechanisms for launching multiple tasks in a
single operation.  The main reason for using such {\em index launches}
is efficiency, as the overhead of starting $n$ tasks with a single
call is much less than launching $n$ separate tasks, and the
difference in performance only grows with $n$.  Thus, when launching
even tens of tasks index launches should be used, provided of course
that the structure of the task launch is sufficiently regular that an
index launch is natural.

\begin{figure}
{\small
\begin{lstlisting}
void top_level_task(const Task *task,
                    const std::vector<PhysicalRegion> &regions,
                    Context ctx,
                    Runtime *runtime)
{
  int points = 50;
  Rect<1> launch_bounds(Point<1>(1),Point<1>(points));
  Domain launch_domain = Domain::from_rect<1>(launch_bounds);
  ArgumentMap producer_arg_map;
  for (int i = 0; i < points; i += 1)
  {
    int subtask_id = 2*i;
    producer_arg_map.set_point(DomainPoint::from_point<1>(Point<1>(i)),
                               TaskArgument(&subtask_id,sizeof(int)));
  }
  IndexLauncher producer_launcher(INDEX_PRODUCER_ID,
                                  launch_domain,
                                  TaskArgument(NULL, 0),
                                  producer_arg_map);
  FutureMap fm = runtime->execute_index_space(ctx, producer_launcher);
  ArgumentMap consumer_arg_map = fm->convert_to_argument_map();
  IndexLauncher consumer_launcher(INDEX_CONSUMER_ID,
                                  launch_domain,
                                  TaskArgument(NULL, 0),
                                  consumer_arg_map);
  runtime->execute_index_space(ctx, consumer_launcher);
}

int subtask_producer(const Task *task,
                     const std::vector<PhysicalRegion> &regions,
                     Context ctx,
                     Runtime *runtime)
{
  int subtask_number = *((const int *)task->local_args);
  printf(``\tProducer subtask %d\n'', subtask_number);
  return subtask_number + 1;
}

void subtask_consumer(const Task *task,
                      const std::vector<PhysicalRegion> &regions,
                      Context ctx,
                      Runtime *runtime)
{
  int subtask_number = *((const int *)task->local_args);
  printf(``\tConsumer subtask %d\n'', subtask_number);
}
\end{lstlisting}
}
\caption{\legionbook{Tasks/indexlaunch/indexlaunch.cc}}
\label{fig:indexlaunch}
\end{figure}


Figure~\ref{fig:indexlaunch} gives an example that uses index
launches. This program is an extension of the example in
Figure~\ref{fig:futures}, but this time instead of launching a single
producer and consumer pair at a time, in Figure~\ref{fig:indexlaunch}
all of the producers are launched in a single Legion runtime call,
followed by another single call to launch all of the consumers.

We now work through this example in detail, as it introduces several
new Legion runtime calls.  First a one dimensional {\tt Rect} called
{\tt launch\_bounds} is created with the points {\tt 1..points}, where
{\tt points} is set to 50. The {\tt Rect} is then converted to a {\tt
  Domain} in preparation for passing it to a Legion call; one task will
be launched for each point in the {\tt Domain}.  Currently, only
{\tt Rects} can be used to define the index space of a task launch---i.e.,
the index space must be a dense 1, 2 or 3 dimensional space.

When launching mutliple tasks simultaneously, we need some way to
describe for each task what argument it should receive.  There are two
kinds of arguments that Legion supports: arguments that are common to
all tasks (i.e., the same value is passed to all the tasks) and
arguments that are specific to a particular task.
Figure~\ref{fig:indexlaunch} illustrates how to pass a (potentially)
different argument to each subtask.  An {\tt ArgumentMap} maps a point
(specifically, a {\tt DomainPoint}) $p$ in the task index space to an
argument for task $p$. In the figure, the {\tt ArgumentMap} maps $p$
to $2p$.  Note that an {\tt ArgumentMap} does not need to name an argument
for every point in the index space.

The procedure for launching a set of tasks is analogous to launching a
single task.  First an {\tt IndexLauncher} is created, which takes
four arguments: the ID of the task to be launched, the index domain
(which determines the number and identity of the tasks launched), a
{\tt TaskArgument} which is passed to all of the tasks, and an {\tt
  ArgumentMap} that gives a mapping from task ID's to the per task
arguments.  Once the {\tt IndexLauncher} is created it can be executed
with a call to {\tt runtime->execute\_index\_space}, which 
takes the parent task's context and the {\tt IndexLauncher} as arguments.

Note that {\tt execute\_task\_space} for the producer tasks returns not
a single {\tt Future}, but a {\tt FutureMap}, which is analagous to
the {\tt ArgumentMap} in that it maps each point in the index
space to a {\tt Future}.  Figure~\ref{fig:indexlaunch} shows one way to
use the {\tt FutureMap} by converting it to an {\tt ArgumentMap} that is fed into
the index launch for the consumer tasks.  Note that the {\tt convert\_to\_argument\_map} call
does not block on the {\tt Future} values, but simply notifies Legion of the dependency
of each point in the index space on a specific {\tt Future}, which is used in the
consumer index launch to provide the same execution semantics as the single task
launches in Figure~\ref{fig:futures}: each consumer subtask runs only after its {\tt FutureMap}
argument is available.

The subtask definitions are straightforward.  Note that the argument specific to the subtask is
in the field {\tt task->local\_args}.  Also note that in the consumer task that 
the argument is not a future, but a fully evaluated {\tt int}.

Finally, while we do not show the code in Figure~\ref{fig:indexlaunch}, the boolean
flags passed in registering the tasks with the Legion runtime are not the same as in
previous examples.  For instance, the registration call
\begin{verbatim}
 Runtime::register_legion_task<subtask_producer>(
   INDEX_PRODUCER_ID, Processor::LOC_PROC, 
   false/* no single launch*/, true/* multiple launch*/);
\end{verbatim}
says that the task {\tt subtask\_producer} with ID {\tt INDEX\_PRODUCER\_ID} is ineligible for
single task launches and must only be used for index task launches.


