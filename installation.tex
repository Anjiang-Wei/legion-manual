
\chapter{Installation}
\label{chap:start}

The Legion homepage is \url{legion.stanford.edu}.  Here you will find
links to everything associated with the project, including a set of
tutorials that are distinct from this manual.  The Legion distribution is at
\url{https://github.com/StanfordLegion/legion}.  The distribution has been
tested on Linux machines and Mac OS X.  To install, in a shell type
\begin{lstlisting}[language=bash]
> cd DIR
> git clone https://github.com/StanfordLegion/legion
\end{lstlisting}
where {\tt DIR} is a directory of your choice.  This command creates 
the directory {\tt DIR/legion}.  To complete the installation,
set the environment variable {\tt LG\_RT\_DIR} to {\tt DIR/legion/runtime}.
For {\tt bash} users, an example {\tt .bashrc} is included in
\legionbook{Installation}.

\section{Regent}

If only the Legion \Cpp\ runtime is desired, there is no need to install Regent, the companion
Legion programming language, and this section can be ignored.  To install Regent, download LLVM
from the following URL
{\small\url{http://llvm.org/releases/3.4.2/clang+llvm-3.4.2-x86_64-apple-darwin10.9.xz}}
and extract it in a directory of your choice:
{\small
\begin{verbatim}
> cd DIR
> tar xf http://llvm.org/releases/3.4.2/clang+llvm-3.4.2-x86_64-apple-darwin10.9.xz
\end{verbatim}
}
You must then put the LLVM {\tt bin} directory in your search {\tt
  PATH} and the LLVM {\tt lib} in yor {\tt DYD\_LIBRARY\_PATH}.  The
example {\tt .bashrc} file in \legionbook{Installation} contains the
necessary commands for {\tt bash} users.
