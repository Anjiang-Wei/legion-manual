\chapter{Regions}
\label{chap:regions}

Regions are the primary abstraction for managing data in Legion.  Futures,
which the examples in Chapter~\ref{chap:tasks} emphasize, are for passing small amounts of data
between tasks. Regions are for holding and processing bulk data.

Because data placement and movement are crucial to performance in modern machines,
Legion provides extensive facilities for managing regions.  These features are a
distinctive aspect of Legion and also probably the most novel and unfamiliar 
to new Legion programmers.  Most programming systems hide the placement,
movement and organization of data; in Legion, these operations are exposed to
the application.

Figure~\ref{fig:lr1} shows a very simple program that
creates a {\em logical region}.  A logical region is a table (or,
equivalently, a relation), with an {\em index space} defining the rows
and a {\em field space} defining the columns. The example
in Figure~\ref{fig:lr1} illustrates a number of points:

\begin{itemize}

\item An {\tt IndexSpace} defines a set of indices for a region.  The {\tt create\_index\_space}
call in this program creates a index space with 100 elements.  Multidimensional index spaces can be
created from multidimensional {\tt Rect}s.

\item Field spaces are created in a manner analogous to index spaces.
  Unlike indices, whose size must be declared, there is a global upper
  bound on the number of fields in a field space (and exceeding this bound will cause
  the Legion runtime to report an error).  This particular
  field space has only a single field {\tt FIELD\_A}.  Note that each field has an associated type, the
  size of which is the first argument to {\tt allocate\_field}.

\item Once the index space and field space are created, they are used to create
a logical region {\tt lr1}.  A second call to {\tt create\_logical\_region}
creates a separate logical region {\tt lr2}.  It is very common to build
multiple logical regions with either the same index space, field space or both.
By providing separate steps for creating the field and index spaces prior to creating
a logical region, application programmers can reuse them in the creation of multiple
regions, thereby making it easier to keep all the regions in synch as the program 
evolves.
\end{itemize}

The logical regions in this example never hold any data.  In
fact, the logical regions consume no space except for their metadata
(number of entries, names of the fields, etc.).  A {\em physical
  instance} of a logical region holds a copy of the actual data for
that region.  The reason for having both concepts, logical region and
physical instance, is that there is not a one-to-one relationship
between logical regions and instances.  It is common, for example, to
have multiple physical instances of the same logical region (i.e.,
multiple copies) distributed around the system in some fashion to
improve read performance.  Because this program does not create any
physical instances, no real computation takes place, either; the
example simply shows how to create, and then destroy, a logical
region.

\begin{figure}
{\small
\begin{lstlisting}
  // create an index space
  Rect<1> rec(Point<1>(0),Point<1>(99));
  IndexSpace is = runtime->create_index_space(ctx,rec);

  // create a field space                          
  FieldSpace fs = runtime->create_field_space(ctx);
  FieldAllocator field_allocator = runtime->create_field_allocator(ctx,fs);
  FieldID fida = field_allocator.allocate_field(sizeof(float), FIELD_A);
  assert(fida == FIELD_A);

  // create two distinct logical regions 
  LogicalRegion lr1 = runtime->create_logical_region(ctx,is,fs);
  LogicalRegion lr2 = runtime->create_logical_region(ctx,is,fs);

  // Clean up.  IndexAllocators and FieldAllocators automatically have their resources reclaimed
  // when they go out of scope. 
  runtime->destroy_logical_region(ctx,lr1);
  runtime->destroy_logical_region(ctx,lr2);
  runtime->destroy_field_space(ctx,fs);
  runtime->destroy_index_space(ctx,is);
\end{lstlisting}
}
\caption{\legionbook{Regions/logicalregions/logicalregions.cc}}
\label{fig:lr1}
\end{figure}

\section{Physical Instances, Region Requirements, Permissions and Accessors}
\label{sec:permissions}

Actually doing something with a logical region requires a {\em
  physical instance}.  The simplest way to create a physical instance
is to pass a logical region to a subtask, as Legion automatically
provides a physical instance to the subtask.  This instance is
guaranteed to be up-to-date, meaning it reflects any changes made to
the region by previous tasks that the subtask depends on.  In the
common case, this means that the results of all previously launched
tasks that updated the region will be reflected in the instance, but
the programmer can specify other semantics if desired; see
Section~\ref{sec:coherence}.

\begin{figure}
{\small
\begin{lstlisting}
  TaskLauncher init_launcher(INIT_TASK_ID, TaskArgument(NULL,0));
  init_launcher.add_region_requirement(RegionRequirement(lr, WRITE_DISCARD, EXCLUSIVE, lr));
  init_launcher.add_field(0, FIELD_A);
  rt->execute_task(ctx, init_launcher);

  TaskLauncher sum_launcher(SUM_TASK_ID, TaskArgument(NULL,0));
  sum_launcher.add_region_requirement(RegionRequirement(lr, READ_ONLY, EXCLUSIVE, lr));
  sum_launcher.add_field(0, FIELD_A);
  rt->execute_task(ctx, sum_launcher);
\end{lstlisting}
}
\caption{Task launches from \legionbook{Regions/physicalregions/physicalregions.cc.}}
\label{fig:permissions}
\end{figure}


Figure~\ref{fig:permissions} shows an excerpt from the top level task in \\
\legionbook{Regions/physicalregions/physicalregions.cc}.  This program is an extension of the
program in Figure~\ref{fig:lr1}---the creation of the (single) logical region is exactly the same as in the 
previous example.  Here we call two tasks that operate on the logical region {\tt lr}. The first
task intializes the elements of the region and the second sums the elements and prints out the results.
As in previous examples, a {\tt TaskLauncher} object describes the task to be called and its non-region arguments,
of which there are none.  When tasks also have region arguments, additional information must be added
to the {\tt TaskLauncher}.
For each region the task will access, a {\em region requirement} must be added to the launcher using the
method {\tt add\_region\_requirement}.  A {\tt RegionRequirement} has four components: 

\begin{itemize}

\item The logical region that will be accessed.

\item A {\em permission}, which indicates how the subtask will
  use the logical region.  In this program, the two tasks have
  different permissions: the initialization task accesses the region
  with permission {\tt WRITE\_DISCARD} (which means it will overwrite
  everything that was previously in the region) and the sum task
  accesses the region with permission {\tt READ\_ONLY}.  Permissions are
  used by the Legion runtime to determine which tasks can run in
  parallel.  For example, if two tasks only read from a region, they
  can execute simultaneously.  Other interesting permissions that we
  will see in future examples are {\tt READ\_WRITE} (the task both
  reads and writes the region), {\tt WRITE} (the task only writes the
  region, but may not update every element as in {\tt
    WRITE\_DISCARD}), and {\tt REDUCE} (the task performs reductions
  to the region).  It is an error to attempt to access a region in a
  manner inconsistent with the permissions, and most such errors can be
  checked by the Legion runtime with appropriate debugging settings.
  The runtime cannot check
  that every element is updated when using permission {\tt
    WRITE\_DISCARD} and failure to do so may result in incorrect
  behavior.

\item A {\em coherence mode}, which indicates what the subtask expects to see from {\em other} tasks that may access the
region simultaneously.  The mode {\tt EXCLUSIVE} means that this subtask must appear to have exclusive access to the region---if
any other tasks do access the region, any changes they make cannot be visible to this subtask. Furthermore, the subtask
must see all updates from previously launched tasks. Other coherence modes that we will discuss are {\tt ATOMIC} and
{\tt SIMULTANEOUS} (see Chapter~\ref{chap:coherence}).

\item Finally, the region requirement names its {\em parent region}.
  We have not yet discussed subregions (see
  Chapter~\ref{chap:partitioning}), so we defer a full explanation of
  this argument.  Suffice it to say that it should either be the
  parent region or, if the region in question has no parent, the
  region itself, as in this example.

\end{itemize}

Finally, each region requirement applies to one or more fields of the region, and the method {\tt add\_field} is
used to record which field(s) each region requirement applies to.
In this example, there is only one region requirement with index 0 (region requirements
are numbered from 0 in the order they are added to the launcher) and a single field {\tt FIELD\_A} that will be
accessed by the subtask.

\begin{figure}
{\small
\begin{lstlisting}
void sum_task(const Task *task,
		    const std::vector<PhysicalRegion> &rgns,
		    Context ctx, Runtime *rt)
{
  const FieldAccessor<READ_ONLY,int,1> fa_a(rgns[0], FIELD_A);
  Rect<1> d = rt->get_index_space_domain(ctx,task->regions[0].region.get_index_space());
  int sum = 0;
  for (PointInRectIterator<1> itr(d); itr(); itr++)
    {
      sum += fa_a[*itr]; 
    }
  printf("The sum of the elements of the region is %d\n",sum);
}
\end{lstlisting}
}
\caption{Region accessors from \legionbook{Regions/physicalregions/physicalregions.cc}.}
\label{fig:accessors}
\end{figure}
We now turn our attention to the two subtasks.  The initialization task and the sum task have very similar
structures, differing only in that the intialization task writes a ``1'' in {\tt FIELD\_A} of every element of the region and
the sum task adds these numbers up and reports the sum.  The sum task is shown in Figure~\ref{fig:accessors}.

When {\tt sum\_task} is called, the Legion runtime guarantees that it
will have access to an up-to-date physical instance of the region {\tt
  lr} reflecting all the changes made by previously launched tasks
that modify the {\tt FIELD\_A} of the region (which in this case is
just the initialization task {\tt init\_task}).  The only new feature
that we need to discuss, then, is how the task accesses the data in {\tt FIELD\_A}.

Access to the fields of a region is done through a {\tt FieldAccessor}.  Accessors in Legion provide a level of indirection
that shields application code from the details of how physical instances  are represented in memory.  Under the hood, 
the Legion runtime chooses among many different representations depending on the circumstances, so this extra level
of abstraction avoids having those details exposed and fixed in application code.  There are several different types of
region accessors provided by Legion.  The {\tt Generic} accessor has 
extensive debugging but  it is also very slow and should never be used in production code.  
The {\tt FieldAccessor} used in Figure~\ref{fig:accessors} does no checking and is much more performant.

In Figure~\ref{fig:accessors}, the field
{\tt FIELD\_A} is named in the creation of a {\tt RegionAccessor} for the first (and only) physical region argument.
Note that the type of the field is also included as part of the construction of the accessor.
The other requirement to access the region is knowledge of the region's index space.  Figure~\ref{fig:accessors}
illustrates how to recover a region's index space from a physical instance of the region using the {\tt get\_index\_space} method.
Since this region has a dense index space, we convert the domain to a rectangle (using the {\tt get\_rect} method).
All that is left, then, is to iterate over all the points of the index space (the rectangle {\tt rect}) and read the
field {\tt FIELD\_A} for each such point in the region using the field accessor {\tt acc}.

\section{Fill Fields}
\label{sec:fill}

It is common to initialize all instances of a particular field in a region to the same value, and so Legion
provides direct support for this idiom.  Figure~\ref{fig:fill} gives an excerpt from an example identical
to the one in Figure~\ref{fig:accessors}, except that the initialization task has been replaced by a call to
the runtime that fills every occurrence of {\tt FIELD\_A} with a default value.

\begin{figure}
{\small
\begin{lstlisting}
LogicalRegion lr = rt->create_logical_region(ctx,is,fs);

int init = 1;
rt->fill_field(ctx,lr,lr,fida,&init,sizeof(init));
\end{lstlisting}
}
\caption{\legionbook{Regions/fillfields/fillfields.cc}}
\label{fig:fill}
\end{figure}
The code in Figure~\ref{fig:fill} uses the Legion runtime method {\tt fill\_field} to initialize every 
occurrence of {\tt FIELD\_A} to 1.  The {\tt fill\_field} method takes six arguments:

\begin{itemize}

\item Like almost all runtime calls, the first argument is the current task's context.

\item The second argument is the region to be initialized.

\item The third argument is the parent region, or the region itself if it has no parent.  The parent region is needed
to ensure that there are sufficient privileges to perform the initialization ({\tt READ\_WRITE} permission
is required).

\item The fourth argument is the ID of the field to be initialized.

\item The fifth argument is a buffer holding the initial value.

\item The sixth argument is the size of the buffer.  
The {\tt fill\_field} call makes a copy of the buffer.

\end{itemize}

The advantage of using {\tt fill\_field} is that the Legion runtime performs the initializaion lazily the next time that
the field is used, which makes the operation less expensive than a normal task call.  Thus, {\tt fill\_field} is preferred
whenever all instances of a field are initialized to the same value.


\section{Inline Launchers}
\label{sec:inlinelaunch}

